\chapter{Conclusions and further work}
% \addcontentsline{toc}{chapter}{Conclusions}
\markboth{Conclusions and further work}{}

In the present thesis, we have described and successfully implemented a mathematical model that simulates the early stages of embryonic organoid development, and used it to simulate observed phenomena. The model describes the movement of each one of the cells in a multicellular aggregate, and the differentiation process that these cells simultaneously go through, coupling both processes. We have discussed every step of the construction of the model in an attempt to convey the reasoning behind each decision.

The \textit{in vitro} procedure studied consists in seeding stem cells into a well, letting them aggregate, and studying the evolution of the cell culture. Under the appropriate conditions, the aggregate eventually resembles an organ-like structure (organoid henceforth). In the model, the initial aggregate is obtained by considering the proliferation of stem cells that divide for a given time, which arises structures as those resulting from the aggregation in the well.

The study focuses on the differentiation kinetics of the aggregate -- the transition of cells from one state to another. Transition can be affected by the surroundings of a cell, so we need an accurate description of the position of the cells. 

We have built the equations of motion for each cell using a centre-based approach where each cell is represented as a sphere, following the models suggested by \cite{Liedekerke_2015} and \cite{Saiz_2020}, accounting for passive forces, active forces and friction forces. We have found that a direct implementation of the initial proposed equations of motion leads to numerical instabilities and unphysiological aggregate configurations. Consequently, we have presented simplifications to the equations, analysed the issues encountered, and determined a final functional model for the motion in which the repulsive and attractive passive forces follow different expressions. We have also described a method for the implementation of active forces modelled as filopodia that generate random cell-cell interactions, based on \cite{Oriola_2022}.

The differentiation process has been modelled as three-state unidirectional sto\-chastic process, where cells transition from the initial state $A$ to $B$, and from state $B$ to the final state $C$. Experimental data from gastruloids built using mouse embryonic stem cells (mESCs) shows that the cells present in the early stages of the formation process can be classified into three states in terms of their gene expression: T+ ($A$), T- ($B$), and further states. Results show that cells in the initial state inhibit the transition of their neighbours, and the role of the initial T- population in early development \parencite{Oriola_2025}. 

We have implemented cell differentiation considering three possible differentiation kinetics: linear, mean field feedback, and cell-cell signalling feedback. Even though the first two can be solved analytically, cell-cell signalling is a process that depends on the ever-changing neighbours of a cell. This simulation offers a new tool for the accurate reproduction of this process, and compares it to the aforementioned simpler models.

The programming of the model has been done using CellBasedModels.jl, the Julia package for agent-based modelling, along with the provided documentation 
(see \cite{Torregrosa_2025}). Using this software, we have been able to simulate in the span of minutes processes that take days to develop in the laboratory. Also, we have contributed to expanding the documentation for this novel framework, providing examples and functions to be extrapolated to future programs.

Finally, we have tested the model using experimental measurements to simulate the cell-cell signalling feedback described in \cite{Oriola_2025}, and have obtained data for the evolution of the proportions over time and depending on the initial proportion of T+ cells. These simulations include varying proliferation rates and simulating the differential adhesion phenomena that lead to the formation of an outer layer in the aggregate. We conclude that the model successfully simulates the early stages of embryonic organoids, and is able to couple differentiation behaviours to the motion of cells in the aggregate.

There are several ways to expand the work presented in this text. Following this study, alternative cell-cell signalling models could be implemented. The model could be easily modified to account for factors that might be relevant for other processes, such as cellular death, more states, or non-unidirectional transitions. Additionally, greater computation power or alternative computing methods could ease the study of computationally expensive simulations, such as proliferation during differentiation and large protrusion strength, and allow computing more realizations. Finally, a deeper statistical analysis could be performed.
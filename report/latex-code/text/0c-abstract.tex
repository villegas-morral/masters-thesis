\thispagestyle{plain}                                   % simple page numbering
\newgeometry{margin=4.25cm, bottom=3cm, top=3cm}        % should be the same as in the acknowledgements
\setstretch{1.2}

\begin{center}
    {\Large\bfseries\sffamily  Abstract}
\end{center}

\vspace{0.5em}

In recent years, embryonic organoids have become an increasingly important tool for studying complex developmental processes that are difficult to access in actual animal embryos. These \textit{in vitro} structures can be generated in a high-throughput manner and have the advantage of mimicking many intricate behaviours observed in real embryos. In this study, we describe an agent-based model designed to simulate the development of embryonic organoids, and implement it using CellBasedModels.jl, the Julia package for multicellular modelling. Our model integrates mechanical interactions with cellular differentiation and accounts for the stochastic nature of both cell-cell interactions and cell fate transitions. We provide a comprehensive overview of the key simplifications made in the mathematical model and offer a numerical justification for these choices. Finally, we use the model to simulate the early stages of gastruloid formation, analysing the roles of cell differentiation and differential adhesion.

\vspace{2em}

\begin{center}
    {\Large\bfseries\sffamily Resum}
\end{center}

\vspace{0.5em}

En els darrers anys, els organoides embrionaris han esdevingut una eina cada cop més important per estudiar processos de la biologia del desenvolupament difícils d'explorar en embrions ani\-mals. Aquestes estructures \textit{in vitro} són fàcilment reproduïbles i te\-nen \linebreak l'avantatge de mimetitzar molts dels comportaments com\-plex\-os observats en embrions reals. En aquest estudi, contruïm un model basat en agents per simular el desenvolupament d'organoides embrionaris i l'im\-ple\-men\-tem uti\-litzant CellBasedModels.jl, el paquet de Julia de mo\-de\-lit\-zació multicel·lular. El nostre model integra les interaccions mecà\-niques amb la diferenciació cel·lular, i té en comp\-te la naturalesa estocàstica tant de les interaccions entre cèl·lules com de la diferenciació cel·lular. Presentem una visió completa de les principals simplificacions realitzades en el model mate\-mà\-tic i oferim una justificació numèrica per aquestes eleccions. Finalment, utilit\-zem el model per simular les primeres etapes de la formació de \textit{gastruloids}, analitzant els rols de la diferenciació ce\-l·lular i l'adhesió diferencial.

\restoregeometry